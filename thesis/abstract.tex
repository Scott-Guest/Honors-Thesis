\begin{abstract}
	To support reasoning about incomplete programs in a principled way, various programming systems have introduced \emph{typed-holes} - placeholder terms which indicate missing syntactic pieces or semantic inconsistencies \cite{GHCHoles, DBLP:journals/jfp/Brady13, DBLP:conf/icfp/Norell13, DBLP:conf/popl/OmarVHAH17}. Ideally, these holes allow every intermediate edit state of a program to be given static or even dynamic meaning, with the aim of enabling simpler and more powerful development tools \cite{DBLP:conf/snapl/OmarVHSGAH17, DBLP:journals/pacmpl/OmarVCH19}. However, current systems are limited in that they only support holes in expressions or types, presenting difficulty when editing binding constructs such as patterns. To resolve this, we propose the development of Peanut: a calculus for pattern matching with holes in patterns, including support for exhaustiveness and redundancy checking in this setting. As well, to ensure its correctness, we propose a mechanization of Peanut's semantics and metatheory in the Agda proof-assistant \cite{norell:thesis}.
\end{abstract}