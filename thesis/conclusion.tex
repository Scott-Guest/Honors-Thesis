\section{Conclusion}\label{sec:conclusion}
As the authors can attest, programming is an oft painful process, full of failures, subtle flaws, breaking edits, and frequently fragile tools. Yet, scattered between these frustrating moments are periods of utter joy, when the stars align, code compiles, and programming tools help the despairing developer with clarifying insights, guiding them to their goal. 

In this paper, we seek to make these joyful moments all the more frequent. We continue the line of research of \cite{DBLP:conf/popl/OmarVHAH17,DBLP:conf/snapl/OmarVHSGAH17}, extending the concept of typed-holes to enable pattern matching while allowing all intermediate edit states to maintain full static and dynamic meaning. By implementing our work into Hazel, we also inch closer to the development of the first fully-featured functional programming language which completely eliminates the \emph{gap problem}. As pattern matching is increasingly a central feature of modern programming languages, we believe the contributions of this paper represent significant progress towards this goal.

In all, we hope for a bright future, with robust support for typed-holes as commonplace as any other editor service, and with meaningless editor states and gaps in editor services entirely abolished. We hope too, that others share this ambition, continuing to bridge the gap between the beautiful purity of type-theoretic semantic foundations and the utmost messiness of real-world program editors. 